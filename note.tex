\documentclass{article}


\usepackage{PRIMEarxiv}

\usepackage[utf8]{inputenc} % allow utf-8 input
\usepackage[T1]{fontenc}    % use 8-bit T1 fonts
\usepackage{hyperref}       % hyperlinks
\usepackage{url}            % simple URL typesetting
\usepackage{booktabs}       % professional-quality tables
\usepackage{amsfonts}       % blackboard math symbols
\usepackage{nicefrac}       % compact symbols for 1/2, etc.
\usepackage{microtype}      % microtypography
\usepackage{lipsum}
\usepackage{fancyhdr}       % header
\usepackage{graphicx}       % graphics
\graphicspath{{media/}}     % organize your images and other figures under media/ folder

%Header
\pagestyle{fancy}
\thispagestyle{empty}
\rhead{ \textit{ }} 

% Update your Headers here
\fancyhead[LO]{Running Title for Header}
% \fancyhead[RE]{Firstauthor and Secondauthor} % Firstauthor et al. if more than 2 - must use \documentclass[twoside]{article}



  
%% Title
\title{A Note of Constrained Exploration in Reinforcement Learning with Optimality Preservation
%%%% Cite as
%%%% Update your official citation here when published 
\thanks{\textit{\underline{Citation}}: 
\textbf{Authors. Title. Pages.... DOI:000000/11111.}} 
}

\author{
  Author1, Author2 \\
  Affiliation \\
  Univ \\
  City\\
  \texttt{\{Author1, Author2\}email@email} \\
  %% examples of more authors
   \And
  Author3 \\
  Affiliation \\
  Univ \\
  City\\
  \texttt{email@email} \\
  %% \AND
  %% Coauthor \\
  %% Affiliation \\
  %% Address \\
  %% \texttt{email} \\
  %% \And
  %% Coauthor \\
  %% Affiliation \\
  %% Address \\
  %% \texttt{email} \\
  %% \And
  %% Coauthor \\
  %% Affiliation \\
  %% Address \\
  %% \texttt{email} \\
}


\begin{document}
\maketitle


\section{Section 2}
$|B|$ \quad the number of elements in set $B$\\
$2^B$ \quad the power set of B \\
$\mathcal{A}$ \quad the alphabet, whose members are characters\\
A string over $\mathcal{A}$ is a sequence of characters each belonging to $\mathcal{A}$\\
$\lambda$ \quad empty string\\
$|s|$ \quad the length of string $s$\\
$\mathcal{A}^{+}$ \quad the set of all possible strings over $\mathcal{A}$\\
$\mathcal{A}^{*} = \lambda \cup \mathcal{A}^{+}$\\
Prefix of string $l$ \quad $l=usv$, $u$ is the prefix of $l$ ($u \leq l$), $s$ is the substring of $l$ ($s\prec l$)\\
$L$ \quad A language $L$ over $\mathcal{A}$ is a subset of $\mathcal{A}^{*}$\\
$\overline{L}$ is the prefix-closure of $L$, $\overline{L}=\{u\in\hat{\mathcal{A}}^{*}|u\leq l \  \mathrm{for} \  \mathrm{some} \   l \in L\}$\\
If $\overline{L}=L$, then $L$ is prefix-closed\\
$\mathcal{L}$ \quad specification language\\
Automaton \quad $(\mathcal{S},\mathcal{A},\delta,\Gamma,s^{\circ},\mathcal{S}^{\bullet})$\\
$\mathcal{S}$ \quad finite set of states\\
$\mathcal{A}$ \quad finite set of events (an alphabet)\\
$\delta$ \quad $\mathcal{S}\times\mathcal{A}\rightarrow\mathcal{S}$ transition function\\
$\Gamma:\mathcal{S}\rightarrow2^{\mathcal{A}}$ \quad active event function, in RL is a set of all actions for which $\delta(s,a)$ is defined\\
$s^{\circ}$ \quad initial state\\
$\mathcal{S}^{\bullet}$ \quad marked states\\
$\delta(s,a)!$ \quad there is a state $s^{\prime} \in \mathcal{S}$ such that $\delta(s,a)=s^{\prime}$\\
reachable of state $s$ \quad there is a string $l\in\mathcal{A}^*$ such that $\delta(s^{\circ},l)=s$\\
co-reachable of state $s$ \quad there is a string $l\in\mathcal{A}^*$ such that $\delta(s,l) \in \mathcal{S}_x^{\bullet}$\\
trim automaton \quad all of its states are both reachable and co-reachable\\
$\mathcal{R}_e[\mathrm{X}]$ \quad reachable part of automaton $\mathrm{X}$\\
$L[\mathrm{X}]$ \quad language generated by $\mathrm{X}$, defined as $L(\mathrm{X}):=\{L\in\mathcal{A}^*|\delta_x(s_x^{\circ},l)!\}$\\
$L_m[\mathrm{X}]$ \quad the language marked by $\mathrm{X}$, all of the strings to the final state \\
If $\mathrm{X}$ is trim, then $L(X)=\overline{L_m(\mathrm{X})}$\\
$u\sim_xv$ \quad equivalent strings, if there exists a state $s\in\mathcal{S}_x$ that $\delta(s^{\circ}_x,u)=\delta(s^{\circ}_x,v)=s$\\
$\mathbb{C}_x(s)$ \quad the equivalent class corresponding to state $s$\\
The product of two automaton: $\mathcal{S}_{z}=\mathcal{S}_{x}\times\mathcal{S}_{y},\mathcal{A}_{z}=\mathcal{A}_{x}\cup\mathcal{A}_{y},\mathcal{S}_{z}^{\bullet}=\mathcal{S}_{x}^{\bullet}\times\mathcal{S}_{y}^{\bullet},\Gamma_{z}((s_{x},s_{y}))=\Gamma_{x}(s_{x})\cap \Gamma_{y}(s_y)$ with $(s_x,s_y)\in\mathcal{S}_z $\\
\begin{equation}
\left.\delta_z((s_x,s_y),a)=\left\{\begin{array}{ll}
\left(\delta_x(s_x,a),\delta_y(s_y,a)\right) & \rm{if} \ a \in \Gamma_x(s_x) \cap \Gamma_y(s_y)\\
\rm{undefined}&\rm{otherwise}\end{array}
\right.\right.
\end{equation}
$X$ is isomorphic to $X||Y$, if $\mathcal{A}_x \subseteq \mathcal{A}_y$ and $L(\mathrm{X})\subseteq L(\mathrm{Y})$\\
$\pi_b$ \quad probability distribution over $\mathcal{A}_x$\\
$L_x^p(l)$ \quad the probabilistic language generated by $(\mathrm{X},\pi_b)$\\
\begin{equation}
\left.L_{x}^p(la)=\left\{\begin{array}{ll}L_{x}^p(l)\pi_b(\delta_x(s_x^{\circ},l),a)& \rm{if}\ \delta_x(s_x^{\circ},l)!\\
L_{x}^p(la)=0& \rm{otherwise}\end{array}\right.\right.
\end{equation}

\section{Section 3}
Unconstrained reinforcement-learning problems modeled by trim automaton$$
\mathrm{G}=(\mathcal{S}_g,\mathcal{A}_g,\delta_g,\Gamma_g,s_g^\circ,\mathcal{S}_g^\bullet)
$$
with properties: $|\mathcal{S}_g|<\infty$, $\delta(s,a)!$ for all $(s,a)\in\mathcal{S}_g\times \mathcal{A}_g$, $s_g^{\circ}$ and $\mathcal{S}_g^{\bullet}$ are known\\
$\pi_b(s,a)>0$ for all $(s,a)\in \mathcal{S}_g \times \Gamma_{g}(s)$\\
Learning process: execute $A$, receive a reward, update the Q\\
$L=\cup_{i=1}^N\bar{l}_{n_i}^{(i)}$ \quad language, which means all action sequence's prefix-closure of all episodes\\
$L_m=\cup_{i=1}^N {l}_{n_i}^{(i)}$ \quad the marked language, which means all sequence of all episodes.\\
If $N \rightarrow \infty$, then $L \rightarrow L(\mathrm{G})$ and $L_m \rightarrow L_m(\mathrm{G})$\\
Robbins-Monro conditions: All state-action pairs be visited infinitely often by the agent during the learning process.\\
Proposition 3.1 non-zero probability of being visited in unconstrained learning process\\

\section{Section 4}
$\Pi$ \quad supervisor\\
$(\mathrm{G},\pi_b,\Pi)$ \quad constrained learning process, use $\mathrm{G}$ to represent $(\mathrm{G},\pi_b)$ or $(\mathrm{G},\pi_b,\Pi)$\\
$\Pi(l)$ \quad actions belong to the admissible action set, based on $l$\\
$\Pi \subseteq \Gamma_g(s)=\mathcal{A}_g$, $\pi_b(s,a)=0$ for all $a \notin \Pi(l)$\\
Definition 4.1. $\Pi: L(\mathrm{G}) \rightarrow2^{\mathcal{A}_g}$\\
Explain: input the action sequence, output an admissible action set, the action set is a subset of the original action set (alphabet)\\
\textit{Extended feedback-control structure}\\
$L(\mathrm{\Pi/\mathrm{G}})$ \quad the language generated by the agent under the supervision of $\Pi$, is defined recursively as:
$$\lambda \in L(\mathrm{\Pi/\mathrm{G}})$$
$$\left.\left[\begin{array}{c}l\in L(\Pi/\mathrm{G})\ \mathrm{and} \ l a \in L(\mathrm{G})\ \mathrm{and}\ a\in\Pi(l)\end{array}\right.\right]\Leftrightarrow la \in L(\Pi/\mathrm{G})$$
$L(\Pi/\mathrm{G})=\overline{L(\Pi/\mathrm{G})}$, $L(\Pi/\mathrm{G})$ is prefix-closed\\
$L(\Pi/\mathrm{G}) \subset L(\mathrm{G})$\\
$\mathcal{L}\subset L(\mathrm{G})$ a specification language, which is regular and prefix-closed\\

\section{Section 5}
Lemma 5.1. An effective $\Pi$ exists for a prefix-closed language $\mathcal{L}\subset L(\mathrm{G})$\\
Proof:
Define a supervisor 
$$\Pi(l)=a\in\mathcal{A}_g|l\in L(\mathrm{G}) {and} la\in\overline{\mathcal{L}}$$
then $l\in L(\Pi/\mathrm{G})$ if and only if $l\in\overline{\mathcal{L}}$\\
Unconstrained RL can be implemented via the product $\mathrm{H}||\mathrm{G}$, $L(\mathrm{H}||\mathrm{G})=L(\Pi/\mathrm{G})$ and $L_m(\mathrm{H}||\mathrm{G})=L_m(\Pi/\mathrm{G})$ \\
Preposition 5.1. A trim automaton H is a realization of the supervisor $\Pi$ if $L(\mathrm{H}) = \mathcal{L}$\\
The probability of a state-action pair$(s,a)$ being visited by the agent can be expressed as
$$
p(a|s)=p_{\pi}(a|l)\cdot p_{\pi}(a|l)\cdot L_{{H||G}}^{p}(l)
$$
where $L_{{H||G}}^{p}(l)$ is the probability of $l\in\mathbb{C}_g(s)\cap L(\Pi/\mathrm{G})$ generated under the supervision of $\Pi$\\
Lemma 5.5. $L_{{H||G}}^{p}(l)>0$\\
Definition 5.2.$\Pi$ is optimality-preserving if $p(a|s)>0$ for all $(s,a)\in(\mathcal{S}_g-\mathcal{S}_g^{\bullet})\times \mathcal{A}_g$\\
Definition 5.3. Let $\mathcal{L} \subset L(\mathrm{G})$. $\mathcal{L}$ covers G if, for any $(s,a)\in(\mathcal{S}_g-\mathcal{S}_g^{\bullet})\times \mathcal{A}_g$, there exists a string $l\in \mathbb{C}_g(s)\cup \mathcal{L}$ such that $la\in \mathcal{L}$\\
Theorem 5.1. An effective supervisor $\Pi$ is optimality-preserving if and only if $\mathcal{L}$ covers G.\\
Proof: based on calculation of the probability of $(s,a)$\\
Visitability of a state (in order to judge whether $\mathcal{L}$ covers $\mathrm{G}$): a state $s_g$ is visitable with respect to $\rm{} M=H||G$ if
$$
\bigcup_{(s_h,s_g)\in\Omega_m(s_g)}\Gamma_m((s_h,s_g))=\mathcal{A}_g
$$
where $\Omega_m:\mathcal{S}_g \rightarrow 2^{\mathcal{S}_m}, \mathcal{S}_m=\mathcal{S}_h \times \mathcal{S}_g$ is a function. \\
Explanation: if we use different action sequence to reach $s_g$, the admission set $\Gamma$ also changes. During training, we may use many sequences to reach $s_g$, if the total union of each $\Gamma$ is the action set, then state $s_g$ is visitable.\\
Lemma 5.3. If $s_g$ is visitable with respect to M, then for any $a\in\mathcal{A}_g$, there exits a state $(s_h,s_g) \in \Omega_{m}(s_g)$ such that $a\in \Gamma_m((s_h,s_g))$\\
Definition 5.5 A automaton G is visitable with respect to M if every state in $(\mathcal{S}_g-\mathcal{S}_g^{\bullet})$ is visitable with respect to M.\\
Theorem 5.2. If G is visitable with respect to M, then $\mathcal{L}$ covers G.\\


\section{Supplementary}
\subsection{Robbins-Monro conditions}
Every state-action pair must have a non-zero probability of being visited by the agent — also known as persistent exploration.
\subsection{Prefix-closed}
Here's an example to illustrate this concept:

Consider the Language: Let's say we have a language L consisting of the following strings over the alphabet {0, 1}: { "", "0", "10", "101", "1011" }. Note that "" represents the empty string.

Check for Prefixes:

The string "0" is in L, and its prefixes are "" (empty string), which is also in L.
The string "10" is in L, and its prefixes are "", "1" (not in L), and "10" (in L).
The string "101" is in L, and its prefixes are "", "1", "10", and "101" (only "10" and "101" are in L).
The string "1011" is in L, and its prefixes are "", "1", "10", "101", and "1011" (only "10", "101", and "1011" are in L).
Assess Prefix-Closed Property:

Since not all prefixes of the strings in L are also in L (for instance, "1" is a prefix of "10" but is not in L), the language L is not prefix-closed.
To contrast, here is an example of a prefix-closed language:

Consider a language M = { "", "0", "00", "000" }.
Every string in M is such that all of its prefixes are also in M. For example, the prefixes of "000" are "", "0", and "00", all of which are in M.
Therefore, M is a prefix-closed language.
\end{document}